\documentclass[titlepage, a4paper, 12pt]{article}
\usepackage[swedish]{babel}
\usepackage[utf8]{inputenc}
\usepackage{verbatim}
\usepackage{fancyhdr}
\usepackage{graphicx}
\usepackage{parskip}

% SourceCode
\usepackage{listings}
\usepackage{color}

% Include pdf with multiple pages ex \includepdf[pages=-, nup=2x2]{filename.pdf}
\usepackage[final]{pdfpages}
% Place figures where they should be
\usepackage{float}

% SourceCode
\definecolor{keywordcolor}{rgb}{0.5,0,0.75}
\lstset{
  inputencoding=utf8,
  language=Java,
  extendedchars=true,
  basicstyle=\scriptsize\ttfamily,
  stringstyle=\color{blue},
  commentstyle=\color{red},
  numbers=left,
  firstnumber=auto,
  numberblanklines=true,
  stepnumber=1,
  showstringspaces=false,
  keywordstyle=\color{keywordcolor}
  % identifierstyle=\color{identifiercolor}
}

% Float for text
\floatstyle{ruled}
\newfloat{kod}{H}{lop}
\floatname{kod}{Kodsnutt}

% vars
\def\title{Genetisk animation; rörelsemönster i 2D mha neuralt nät och genetisk algoritm}
\def\preTitle{}
\def\kurs{Emergenta system, 5DV017}

\def\namn{Andreas Jakobsson}
\def\mail{dit06ajs@cs.umu.se}

\def\namnTva{Anton Johansson}
\def\mailTva{dit06ajn@cs.umu.se}

\def\namnTre{Erik Rönnberg}
\def\mailTre{dit06erg@cs.umu.se}

\def\namnFyra{Ludvig Widman}
\def\mailFyra{dit06lwn@cs.umu.se}

\def\namnFem{Ragnar Asplund}
\def\mailFem{dit04rad@cs.umu.se}

\def\pathtocode{$\sim$dit06ajn/edu/emergenta-system/projektet/src}

\def\handledareEtt{Jonny Pettersson, jonny@cs.umu.se}
\def\handledareTva{Anders Broberg, bopspe@cs.umu.se}

\def\inst{datavetenskap}
\def\dokumentTyp{Projektrapport}

\begin{document}
\begin{titlepage}
  \thispagestyle{empty}
  \begin{small}
    \begin{tabular}{@{}p{\textwidth}@{}}
      UMEÅ UNIVERSITET \hfill \today \\
      Institutionen för \inst \\
      \dokumentTyp \\
    \end{tabular}
  \end{small}
  \vspace{10mm}
  \begin{center}
    \LARGE{\preTitle} \\
    \huge{\textbf{\kurs}} \\
    \vspace{10mm}
    \LARGE{\title} \\
    \vspace{15mm}
    \begin{large}
      \namn, \mail \\
      \namnTva, \mailTva\\
      \namnTre, \mailTre\\
      \namnFyra, \mailFyra\\
      \namnFem, \mailFem\\
      \texttt{\pathtocode}
    \end{large}
    \vfill
    \large{\textbf{Handledare}}\\
    \mbox{\large{\handledareEtt}}
    \mbox{\large{\handledareTva}}
  \end{center}
\end{titlepage}

\newpage
\mbox{}
\vspace{70mm}
\begin{center}
% Dedication goes here
\end{center}
\thispagestyle{empty}
\newpage

\pagestyle{fancy}
\rhead{\today}
\lhead{\footnotesize{\mail, \mailTva\\\mailTre, \mailFyra, \mailFem}}
\chead{}
\lfoot{}
\cfoot{}
\rfoot{}

\begin{abstract}
% TODO: Sammanfattning/abstract

\end{abstract}

\cleardoublepage
\newpage
\tableofcontents
\cleardoublepage

\cfoot{\thepage}
\pagenumbering{arabic}

\section{Introduktion}
% Introduktion till ämnet, inklusive översikt av tidigare studier
% \cite{flake}

\subsection{Karl Sims}
Karl Sims utforskar i Evolvig Virtual Creatures både hur varelsers form och deras beteenden kan utvecklas med hjälp av genetiska algoritmer (GA). Varelserna har sensorer och effektorer som är kopplade till in och utnoder i ett neuralat nät. 

I sina försök använder Sims en GA för att optimera både struktur och vikter i det neurala nätet. Det nät han använder har i varje nod en av många matematiska funktioner; exempelvis sinus, summa, produkt, max, osv. 


\section{Metodbeskrivning}
% Metodbeskrivning. Ska beskrivas så utförligt att läsaren ska kunna
% reproducera ert arbete


Systemöversikt
	GA
		Genetisk algoritm
		Genotypen kodar för vikter i nätet
		Beräknar fitness baserat på data från siumleringen

	Brain
		Neuralt nät
		Styr varelsen
		Får sensordata från simuleringen
		Ger instruktioner till effektorerna

	Simulring
		Phys2d
		En varelse
			Mask
			Leder med begränsade maxvinklar
			En sensor i varje led
		Varelsen rör sig på en yta
		Eventuellt hinder


\section{Resultat}
% Resultat innehållande sammanställning/analys av tester
\section{Diskussion}
% Diskussion av resultatet, koppling till tidigare studier

\bibliographystyle{alpha}
\bibliography{books}

\newpage
\appendix
\pagenumbering{roman}
\section{Källkod}\label{sec:kallkod}
Härefter följer utskrifter från källkoden och andra filer som hör till
denna laboration.

\subsection{Flocking.nlogo}\label{app:Flocking.nlogo}
\begin{footnotesize}
  \verbatiminput{../src/Flocking.nlogo}
\end{footnotesize}
\end{document}
